\documentclass{article}

\title{Android网络防火墙研究与实现综述}

\begin{document}

\section{前言}

腾讯移动安全实验室发布了2016年手机安全报告,报告显示2016年Android病毒包共增加2341.8万,同比增长40.20\%。2016年Android手机病毒感染用户数同比增长62.43\%,感染用户数总量达到5亿人次,达到历年新高。其中资费消耗类占比高达84.24\%,在2016年手机病毒类型中排名第一。
资费消耗类的病毒是指的在未经用户授权的情况下,通过频繁连接网络、发送短信等方式,导致用户资费损失。这部分病毒往往帮助一些广告商提高App装机量或点击率进行恶意推广,不断联网下载,消耗用户流量。

针对资源消耗类病毒,网络防火墙可以启动有效的防范和遏制的作用,将不可信应用的网络权限进行限制,对可疑应用的数据通信进行监控,是遏制这类病毒的最重要手段。此外,对于远程控制、隐私窃取等类型的软件,限制网络通信也是有效可行的重要举措。

\section{Android简介}


\subsection{Android系统架构}

Android操作系统的核心是Linux内核的一个分支,但也有很多的修改和扩充。其中主要包括:去除了X Windows System,不支持GNU库。另外主要增加了两个部分,Binder IPC和Wakelock管理。
\todo{Android架构仍有扩充空间}

Android系统的架构如图所示,分为四层,自上而下分别是应用程序(Applications),应用程序框架(Application Frameworks),系统运行库与Android运行环境(Libraries \& Android Runtime),Linux内核(Linux Kernel)。
\begin{itemize}
\item 应用程序层由各种应用软件组成,如短信,浏览器,地图等,这些应用可以是系统自带,也可以是从应用市场下载。
\item 应用程序框架层提供了应用程序运行需要的各种API和系统服务,主要由Java编写。开发者开发Android应用需要对这一层由较多的了解。它提供了用于实现图形界面的视图,活动管理器,窗口管理器,内容提供者,资源管理器,通知管理器,包管理器,等。
\item Android运行库是为Android系统中各种组建运行而提供的C/C++库的合集,用以实现底层的图形绘制,媒体播放录制,Web页面渲染,3G图形,SQLite数据库等功能。而Android运行时则是Android虚拟机的实现,Android 5.0以前是Dalvik虚拟机,5.0以后是ART虚拟机,用以执行应用软件中的字节码。
\item Linux内核为Android提供硬件驱动,内存管理,进程眼里,网络协议栈和安全体系等实现。同时,Android加入了针对移动设备优化的电源管理和Binder机IPC机制。
\end{itemize}
https://hit-alibaba.github.io/interview/Android/basic/Android-Arch.html

\subsection{安卓安全机制}

( 1) 权限机制
Android 应用框架层提供了限制组件间访问的强制访问控 制机制,系统中定义了一系列安全操作相关的权限标签,应用需 要在配置文件( Manifest. xml) 中利用这些标签声明自己所需的 权限,当用户同意授权后,该应用下属的所有组件将会继承应用 声明的所有权限。同时,组件也可以利用权限标签限制能够与 其交互的组件范围。如图 1 所示,组件 CC1 要求 P1 权限,而应 用程序 B 被用户授予了权限 P1, 因此 CB1 继承 P1 权限并因而 具有访问 CC1 的能力。组件 CA1 则因为不具备 P1 权限而无法 访问 CC1。

Android 支持普通( normal) 、 dangerous) 、 signa-危险( 签名( ture) 以及签名或系统( signatureOrSystem) 四种权限保护级别。 危险级别的权限在应用程序安装时会在屏幕上列出,而普通级 别的权限是隐藏在折叠目录或屏幕上的。签名级别的权限只有 在请求权限的应用程序与声明权限的程序是用相同的证书签名 时才被授权

( 2) 隔离机制

Android 是基于 Linux 内核的,因此每个应用程序运行在自 己的 Linux 进程中。通常每个应用程序被分配唯一的 Linux 用 户 ID, 因此 Linux 的自主访问控制机制保证应用程序的文件对 Android 于其它应用程序是不可见的。特别的, 应用程序运行在 Dalvik 虚拟机中,因此与其它应用程序的代码是隔离的。

\subsection{安全缺陷}

可以说, 在诞生之初就拥有良好的安全机制。 首 先,它通过继承 Linux 2. 6 内核的安全机制实现系统安全; 其次, 又通过自身的 permission 机制实现数据安全。 但面对浩瀚复杂 Android 的安全环境, 原生安全机制是远远不够的,更何况其本 身还存在诸多问题。 Android 原生安全机制缺陷主要体现在以 下几个方面:

( 1) Android 通过 Package Installer 在应用安装时检查应用 所需的权限,并提示用户,由用户做出抉择。 但是,如果用户想 使用该应用的功能,就必须同意应用申请的全部权限,否则 Package Installer 将拒绝安装该应用。

( 2) Android 没有对已授予应用的权限的使用范围进行限 制。比如应用申请了读联系人数据的权限,那么它就可以读取
全部的联系人数据, 而不能根据联系人分组区别对待。 ( 3) 由于所有权限都是在安装时进行检查并授予的,所以 资源的访问不能根据用户的位置、时间等不同而进行动态的限 制。比如用户需参加一个秘密会议,那么此时用户应该可以设 置应用不能访问摄像头、 GPS、 麦克风等资源,但传统的 Android 安全机制是做不到这一点的。

( 4) 用户撤销已授予某个应用权限的唯一方法就是卸载该 应用。

以上安全缺陷不仅降低了 Android 平台的可用性,同时导 致平台容易遭受应用层权限提升攻击, 具体又可分为混淆代理 人攻击和共谋。

[3] 混淆代理人攻击 是指恶意程序利用其它程序的未保护 接口来间接获得特权功能,普遍存在于 Android 缺省程序( 如电 [4, 10,11] 话、闹钟、音乐和设置程序) 以及第三方程序中 。 如图 1 所示,应用程序 A 没有 P1 权限,因此 CA1 无法访问被 P1 权限 所保护的组件 CC1。 然而,由于组件 CB1 提供了未被 P1 保护的 开放接口,因此 CA1 可以利用 CB1 间接地使用 CC1 的功能,这 样就形成了混淆代理人攻击, Re-delega-也被称作权限机制的 tion 问题 [11] 。ASE ( Android Scripting Environment) 程序具有发 [3 ] 送消息、打电话、读联系人、访问照相机等权限, Lucas Davi 就 描述了利用 ASE 程序的漏洞进行的特权提升攻击。

恶意程序可以通过共谋来合并权限,从而执行超出各自特 [4] Soundcomber 权的动作。例如, 通过拥有录音权限的程序和拥 有互联网权限的程序共谋, 从通话过程中捕获信用卡号并泄漏 给远程敌手。共谋程序可以直接通信,也可以利用 Android 核 心系统组件中的公开信道或隐蔽信道,甚至内核控制的信道来 通信。




Android系统自带的通信权限管理仅仅允许对特定应用设置是否允许链接网络,无法对Wifi网络和蜂窝数据网络区分设置,也不支持多种条件下的自定义规则,如灭屏和亮屏,特定网络,限制时段等条件,而且,不支持按协议过滤,按源目的地址过滤等基本防火墙功能。所以我们有必要开发一款Android网络防火墙来完善这些功能,满足用户对Android网络管理更高级的需求。
所以在目前Android用户数量持续增长,资费消耗病毒日渐猖獗,而Android系统网络功能不全的情况下,所以我们提出了设计并实现一个基于Android平台的网络防火墙的课题,旨在通过对Android系统网络通信的监控和管理的实现,研究Android网络安全,设计一个可用的Android网络防火墙,改善Android系统的安全性。



\section{研究现状}
网络防火墙即可以利用Android的内核特性进行实现,也可以建立在Android系统API之上进行实现。


Android内核是linux内核的一种分支,也提供netfilter和iptable,可以参考Linux内核防火墙的实现方式实现Android防火墙。

\subsection{Netfilter防火墙}

https://www.ibm.com/developerworks/cn/linux/network/l-netip/

Netfilter是Linux 2.4.x之后版本提供的包过滤的框架,它在TCP/IP协议栈中提供了相应的Hook,允许我们对网络数据包进行过滤,丢弃,修改等操作。通过netfilter中的一些函数即可实现防火墙应用。而iptables是linux提供的一个基于netfilter工具,使用户可以对整个操作系统收发的数据包就行拦截修改拒绝等操作。
以IPv4为例,netfilter在IPv4协议栈中定义了五个hook,在这五个点使用nf\_hook\_ops结构注册hook,过滤经过该点的所有包,nf\_hook\_ops定义如下:

\begin{lstlisting}[language=c]
struct nf_hook_ops
{
    struct list_head list;
    nf_hookfn *hook;
    int pf;
    int hooknum;
    int priority;
};
\end{lstlisting}

nf\_hook\_ops中的nf\_hookfn就是hook函数,实现一个nf\_hookfn函数,在该函数中实现对包的读取解析并过滤,返回NF\_ACCEP,NF\_DROP等返回值,即可实现对数据包的操作。

\subsection{VPN防火墙}

Android系统本身提供了制作VPN应用的API,允许第三方应用向系统提供VPN,系统负责将网络流量转发到VPN应用,VPN应用可以建立隧道实现VPN服务。而在这个过程中,VPN应用可以对网络流量进行分析和过滤,实现防火墙的功能。

\subsection{方案对比}
netfilter/iptable需要调用netfilter api或执行iptable,这都必须拥有root权限才有可能实现。Root本身就是利用漏洞才得以实现的,并且root之后的手机毫无安全性可言,与我们做防火墙提高安全性的初衷相悖。

VPN方案由于使用了Android官方提供的公开API,可以不需要Root权限运行,只需要用户在第一次启动VPN时授予许可,也没有给系统带来额外的安全风险。其缺陷在于VPN服务只允许启动一个,将会导致用户使用防火墙后无法使用其他VPN。

以加强Android安全作为出发点,通过对两种方案的对比,我们最后决定采用VPN方案实现Android防火墙。

\section{趋势与展望}

Android网络防火墙除了是解决网络通信相关恶意软件的重要手段,在最近的研究中,也开始更多将注意力放在里将传统防火墙和其他安全技术相结合的方案。

在最新的研究中,研究者往往采用多种其他技术实现对应用程序的可信度进行评估,防火墙成为利用评估结果,实现对不可信应用程序或异常行为的组织和控制的手段。如
Najim Ammari* , Almokhtar Ait El Mrabti, Anas Abou El Kalam, Abdellah Ait Ouahman等人提出了一种解决敏感数据泄漏的防火墙,他们提出的方案利用数据挖掘对大量应用程序二进制文件进行静态分析,通过对敏感操作的钩子代理来对应用程序的危险程度进行动态分析,最后分析结果作为一个关键参数,用于决策防火墙的行为。与传统防火墙不同的是,他们提出的防火墙引入的信任机制。最后结合可信数据和分析数据进行决策。


而还有一部分研究,不仅将防火墙作为最终控制和组织威胁的手段,而开始将防火墙作为一种识别威胁的信息源。

Tendai Munyaradzi Marengereke和K. Sornalakshmi的提出的一个基于云的Android安全监控的收集系统,实现对异常网络流量的监控,响应和处理,其根本就是在于实现网络防火墙流量监控和网络管理,防火墙即是对威胁进行处理和阻止的主体,也是在发现威胁前,进行信息收集的主体。




\section{结语}

在目前Android系统网络模型仍然存在缺陷的情况下,网络防火墙是解决恶意应用恶意网络行为的有效手段。网络防火墙是控制网络行为的关键,也是收集应用程序网络行为,进行应用程序网络行为分析的有效措施。开发和实现网络防火墙,是对安卓网络权限管理模型的一个补充,也是基于网络防火墙的安全系统研究的基石。

\end{document}
\endinput